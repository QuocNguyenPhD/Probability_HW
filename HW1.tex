\documentclass[10pt, letterpaper,]{article}
\usepackage[top=0.25in, bottom=0.25in, left=0.25in,right=0.25in]{geometry}
\usepackage{dsfont, amsmath, amssymb, amsfonts, float, enumerate, mathtools, esvect}
\usepackage{extarrows}
\usepackage{epstopdf}
\usepackage{xspace}
\usepackage[shortlabels]{enumitem}

\pagestyle{empty}
\parindent 0px
\title{Probability Homework 1}
\author{Anshi Gupta, Q Nguyen, Vera Schroeder}
\date{09/11/2023}

\begin{document}
\maketitle
	\section{Computational Project}
		\subsection{Task 1}
		\subsection{Task 2}
			\subsubsection{2.1}
			\subsubsection{2.2}
		\subsection{Task 3}
			\subsubsection{3.1}
			\subsubsection{3.2}
		\subsection{Task 4}
			\subsubsection{4.1}
			\subsubsection{4.2}
			\subsubsection{4.3}
			\subsubsection{4.4}
			
			
			
	\section{Written Problem Solving} The random lifetime $T$ (in hours) of the light bulb in an overhead projector follows an exponential distribution with mean $= M$  hours. During a normal workweek, the projector is used for a random number $N$ of lectures lasting exactly one hour each. The random variable $N$ has a Poisson distribution with mean $= K$. \\
	We recall:
	$$ T \sim \operatorname{Exp}\left[\lambda=\frac{1}{M}\right] \text{ has PDF } f^{T}(t)=P(T=t)=\frac{1}{M} e^{-\frac{t}{M}} \text {, for } t>0 $$
	
	$$ N \sim \operatorname{Poisson} \left[ \lambda = K \right] \text{ has PDF } f^{N}(n)=P(N=n)=\frac{k^{n}}{n !} e^{-k} \text{, for } n=0,1, \ldots$$
		\subsection{Problem 1} Compute the conditional probability $P(T > N | N=n) = P(T>n | N=n)$ for any integer $n=0, 1, 2 ...$ and use the result to compute the probability $P(T > N)$ as an explicit function $u(M,K)$.\\
		
	 	$$
	 	\begin{aligned}
	 	P(T>N \mid N=n)&=P(T>n)=1-P(T \leq n)\\
	 	& =1-\int_{0}^{n} \frac{1}{M} e^{-\frac{t}{M}} d t=1-\int_{-\frac{n}{M}}^{0} e^{x} d x \\
	 	& =1-\left[e^{x}\right]_{-\frac{n}{M}}^{0} \\
	 	& \boxed{ =e^{-\frac{n}{M}}} 
	 	\end{aligned}
	 	$$
		\subsection{Problem 2} use (1) to compute the probability $p(M,K)$ that a projector with a newly installed lightbulb will actually last for a whole  normal  workweek without changing the light bulb \\
		$$
		\begin{aligned}
		p(M, K)&=P(T \geq n) \\
		&=\sum_{n=0}^{\infty} P(T>N \mid N=n) \cdot P(N=n) \\
		&=\sum_{n=0}^{\infty} e^{\frac{-n}{M}} \cdot \frac{k^{n}}{n !} e^{-k} =e^{-k} \sum_{n=0}^{\infty} \frac{\left[e^{-\frac{1}{m}} k\right]^{n}}{n !}\\
		&=e^{-k} \cdot e^{\left[k e^{-1 / n}\right]} =e^{-k+k e^{-1 / n}}\\
		&\boxed{=e^{k\left(e^{-1 / n}-1\right)}}
		\end{aligned}
		$$
		\subsection{Problem 3} Consider that $M$ is imposed by the light bulb commercial brand , and indicate how  to compute $g(M)$ (either by a formula or by numerical computation) such that $K \leq  g(M)$ will force $p(M,K) \geq 0.95$. \\
		$K \leq g(M) \Rightarrow p(M, K) \geq 0.95$
		
		$$
		\begin{aligned}
		e^{k\left(e^{-1 / M}-1\right)}&=0.95 \\
		k\left(e^{-1 / M}-1\right)&=\ln (0.95) \\
		k&= \boxed{\frac{\ln (0.95)}{e^{-1 / M}-1}=g(M)}
		\end{aligned}
		$$
		\subsection{Problem 4} Restrictions on lectures scheduling have succeeded to impose $K = g(M)$. Compute the probability that a projector with a newly installed light bulb will actually last for 4 successive   workweeks without changing the light bulb. \\
		
		Let $X$ be the number of lectures in 4 successive workweeks $X=N_{1}+N_{2}+N_{3}+N_{4}$, where each $N_{i}$ are identically and independently distributed $ \sim \operatorname{Poisson}[\lambda=k]$\\
		
		Then, $X \sim \operatorname{Poisson}[\lambda=4k]$ with the PDF $f^{X}(x)=P(X=x)=\frac{(4k)^{x}}{x !} e^{-(4k)} \text{, for } n=0,1, \ldots $
		$$\begin{aligned}
			P(T>X)&=\sum_{x=0}^{\infty} P(T>X \mid X=x) \cdot P(X=x)\\
			&=\sum_{x=0}^{\infty} e^{-\frac{x}{M}} \cdot \frac{(4 k)^{x}}{x !} e^{-4 k}\\
			&=e^{-4 k} \sum_{x=0}^{\infty} \frac{\left(4 k e^{-1 / n}\right)^{x}}{x !}\\
			&=e^{-4 k} e^{4 k e^{-1 / n}}=e^{4 k\left(e^{-1 / m}-1\right)}\\
		\end{aligned}$$
		Now, we substitute in the value for $k$ we found in Problem 3
		$$\begin{aligned}
		P(T>X)&=e^{4 k\left(e^{-1 / m}-1\right)}\\
		&=\exp \left[4\left(\frac{\ln (0.95)}{e^{-1 / M}-1}\right)\left(e^{-1 / M}-1\right)\right]\\
		&=e^{4 \ln (0.95)}=(0.95)^{4}\\
		&= \boxed{0.81450625}
		\end{aligned}$$
\end{document}